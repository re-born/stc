\documentclass{../style/sig-alternate}

\makeatletter
 \let\@copyrightspace\relax
 \makeatother

\begin{document}

\title{Ttile}

\numberofauthors{5}
\author{
% You can go ahead and credit any number of authors here,
% e.g. one 'row of three' or two rows (consisting of one row of three
% and a second row of one, two or three).
%
% The command \alignauthor (no curly braces needed) should
% precede each author name, affiliation/snail-mail address and
% e-mail address. Additionally, tag each line of
% affiliation/address with \affaddr, and tag the
% e-mail address with \email.
%
% 1st. author
\alignauthor
Denawa\\
       \affaddr{Disavowed, U.S.A}\\
       \email{toma@cruise.com}
% 2nd. author
\alignauthor
Sano\\
       \affaddr{Disavowed, U.S.A}\\
       \email{tomb@cruise.com}
% 3rd. author
\alignauthor
Kadotami\\
       \affaddr{Disavowed, U.S.A}\\
       \email{tomc@cruise.com}
\and  % use '\and' if you need 'another row' of author names
% 4th. author
\alignauthor
Kato\\
       \affaddr{Waseda University}\\
       \email{sow@suou.waseda.jp}
% 5th. author
\alignauthor
Sakai\\
       \affaddr{Disavowed, U.S.A}\\
       \email{tome@cruise.com}
}

\maketitle

\begin{abstract}
Realizing conversation in natural language between human and computer is one of the great problems. It involves understanding natural language and learning and storing knowledges. On the other hand, text-based conversations were made recently over SNS services, Twitter and Facebook, for instance.
Thanks to these services, IR(Information Retrieve)-oriented approach is expected as a method to attack to the problem.
 In this research, I made a system which learns conversation model using a popular machine learning method, Error Back Propagation. Using the model, given a Post-tweet, the system bring a recommended tweet as a reply to it.
\end{abstract}

\section*{Team Name}
SLSTC

\section*{Subtasks}
%Climb the Dubai Tower (Chinese)\\
%Climb the Dubai Tower (English)\\
%Climb the Dubai Tower (Japanese)
% Climb the Dubai Tower (Chinese, English, Japanese)
???

\keywords{keywords}

\section{Introduction}

Short Text Conversation Task (STC Task)[1] is the problem of suggesting appropriate reply to a given input sentence. This is a high level technique of natural language understanding and generating reply sentence. Recently, the great progress of SNS service made it available to use a great deal of conversation data. 

Short Text Conversation Task (STC Task)[1] is the problem of suggesting appropriate reply to a given input sentence. This is a high level technique of natural language understanding and generating reply sentence. Recently, the great progress of SNS service made it available to use a great deal of conversation data. 

Short Text Conversation Task (STC Task)[1] is the problem of suggesting appropriate reply to a given input sentence. This is a high level technique of natural language understanding and generating reply sentence. Recently, the great progress of SNS service made it available to use a great deal of conversation data. 

Short Text Conversation Task (STC Task)[1] is the problem of suggesting appropriate reply to a given input sentence. This is a high level technique of natural language understanding and generating reply sentence. Recently, the great progress of SNS service made it available to use a great deal of conversation data. 

% SLSTC (The Sakai Laboratory, Waseda University) participated in the
% Japanese subtask of the STC task.
% This paper briefly describes our approaches,
% and reports on the official results.
% 
% Table XXX  shows the list of runs that we submitted to the STC Japanese subtask.
% In Section~\ref{s:methods}, we describe the algorithms we employed
% to generate these runs.
% In Section~\ref{s:results}, we discuss the official results of our runs.
% Finally, in Section~\ref{s:conclusions}, we conclude this paper
% and lists up future work items.

\section{Methods}
\label{s:methods}

\subsection{[Runの名前]}

[Peterの手法]

\subsection{[Runの名前]}

[Mチームの手法1]

\subsection{[Runの名前]}

[Mチームの手法2]

\section{Official Results and Discussions}\label{s:results}

% 結果は4月に執筆
% \section{Conclusions and Future Work}
% \label{s:conclusions}
% とりあえず空白
% Mission accomplished.

\bibliographystyle{abbrv}

\bibliography{../bib/references}


\end{document}
